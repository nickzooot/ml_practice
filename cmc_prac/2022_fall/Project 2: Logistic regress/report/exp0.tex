\subsection{Эксперимент 0. Анализ  аналитической и разностной формулы градиента  для логистической регрессии.}
\subsubsection{Дизайн эксперимента}
В ходе данного эксперимента производилась разностная проверка градиента, а именно, использовалось следующее:
	$$
		[\nabla f(x)]_i \approx \frac{f(x + \varepsilon e_i) - f(x)}{\varepsilon},
	$$
	где $e_i = (0,\dots,0,1,0,\dots,0)$; $i$ --  индекс, в котором стоит единица, а
	$\varepsilon \sim \sqrt{\varepsilon_{mach}}$, где $\varepsilon_{mach}$ - машинная точность ($\approx\!10^{-16}$). Причем в нашей задаче в качестве
	$f(\omega)$ будет $Q(X, y, \omega)$. В качестве параметров($X$, $y$) и переменной($w$) брались следующие значения:
	\begin{itemize}
		\item $X \sim \mathcal{U}(-100, 100)$, $X \in \mathbb{R}^{m \times n}$, где $m$ и $n$ случайные целые числа с равномерным распределением от 3 до 7.
		\item $y \in \mathbb{R}^{m \times 1}$ -  случайный вектор из -1 и 1 
		\item $w \sim \mathcal{U}(-5, 5)$, $w \in \mathbb{R}^{n \times 1}$
	\end{itemize}
	Вектор $\omega$ случайным образом брался 10 раз, считался по нему теоретический градиент и численный. Считалась средняя абсолютная ошибка по координатам вектора $\omega$, а далее считалась средняя  ошибка по всем 10 измерениям.
\subsubsection{Результаты эксперимента}
В ходе эксперимента была получена следующая средняя ошибка: $7.7~\times~10^{-7}$
\subsubsection{Выводы эксперимента}
Таким образом, во-первых, была проверена правильность посчитанной аналитической формулы градиента, поскольку средняя ошибка очень мала. Во-вторых, было эвристически показано, что при условии правильности аналитической формулы градиента разностная формула хорошо приближает значение градиента в точке(в нашем примере с точностью до 5-го порядка)
